\documentclass[]{scrartcl}
\usepackage{hyperref}

\begin{document}
\abstract{The following are a summary of relevant sparse linear solvers available from PETSc (I included what I thought we will use based on what I learned from the thesis). \\ In short, the types of matrices are:
	\begin{itemize}
		\item sparse matrices
		\item block sparse matrices
		\item sequential sparse matrices, based on compressed sparse row format
		\item sequential block sparse matrices, based on block sparse compressed row format
	\end{itemize} 
\section{Krylov Methods: Conjugate Gradients (KSPCG)}
\url{http://www.mcs.anl.gov/petsc/petsc-current/docs/manualpages/KSP/KSPCG.html} \\
The Preconditioned Conjugate Gradient (PCG) iterative method.
\begin{itemize}
	\item Requires both the matrix and preconditioner to be symmetric positive (or negative) (semi) definite.
	\item Only left preconditioning is supported.
	\item Parallel and complex.
	\item For complex numbers there are two different CG methods, one for Hermitian symmetric matrices and one for non-Hermitian symmetric matrices.
\end{itemize}
\section{Krylov Methods: GMRES (KSPGMRES)}
\url{http://www.mcs.anl.gov/petsc/petsc-current/docs/manualpages/KSP/KSPGMRES.html} \\
Left and right preconditioning are supported, but not symmetric preconditioning.
\section{CG for Least Squares (KSPCGLS)}
\url{http://www.mcs.anl.gov/petsc/petsc-current/docs/manualpages/KSP/KSPCGLS.html}
Conjugate Gradient method for Least-Squares problems
\begin{itemize}
	\item Supports non-square (rectangular) matrices.
	\item Parallel and complex.
\end{itemize}
\section{Preconditioner: Jacobi (PCJACOBI)}
\url{http://www.mcs.anl.gov/petsc/petsc-current/docs/manualpages/PC/PCJACOBI.html}
Diagonal scaling preconditioning. \\
Parallel and complex. \\
Matrix Types:
\begin{enumerate}
	\item MATAIJ = "aij" - A matrix type to be used for sparse matrices.
	\item MATBAIJ = "baij" - A matrix type to be used for block sparse matrices.
	\item MATSBAIJ = "sbaij" - A matrix type to be used for symmetric block sparse matrices.
	\item MATDENSE = "dense" - A matrix type to be used for dense matrices.
\end{enumerate}
\section{Preconditioner: Point Block Jacobi (PCPBJACOBI)}
\url{http://www.mcs.anl.gov/petsc/petsc-current/docs/manualpages/PC/PCPBJACOBI.html} \\
Uses dense LU factorization with partial pivoting to invert the blocks; if a zero pivot is detected a PETSc error is generated. \\
Parallel and complex.\\
Matrix Types:
\begin{enumerate}
	\item AIJ
	\item BAIJ
\end{enumerate}
\section{Preconditioner: Block Jacobi (PCBJACOBI)}
\url{http://www.mcs.anl.gov/petsc/petsc-current/docs/manualpages/PC/PCBJACOBI.html} \\
Use block Jacobi preconditioning, each block is (approximately) solved with its own KSP object. \\
Each processor can have one or more blocks, or a single block can be shared by several processes. Defaults to one block per processor. \\
Parallel and complex.\\
Matrix Types:
\begin{enumerate}
	\item AIJ
	\item BAIJ
	\item SBAIJ
\end{enumerate}
\section{Preconditioner: ILU (PCILU)}
\url{http://www.mcs.anl.gov/petsc/petsc-current/docs/manualpages/PC/PCILU.html} \\
For BAIJ matrices this implements a point block ILU. \\
Parallel and complex.\\
Matrix Types:
\begin{enumerate}
	\item MATSEQAIJ = "seqaij" - A matrix type to be used for sequential sparse matrices, based on compressed sparse row format.
	\item MATSEQBAIJ = "seqbaij" - A matrix type to be used for sequential block sparse matrices, based on block sparse compressed row format.
\end{enumerate}
\section{Direct Solvers: Cholesky (PCCHOLESKY)}
\url{http://www.mcs.anl.gov/petsc/petsc-current/docs/manualpages/PC/PCCHOLESKY.html} \\
Uses a direct solver, based on Cholesky factorization, as a preconditioner. \\ 
Usually this will compute an "exact" solution in one iteration. \\
Matrix Types:
\begin{enumerate}
	\item seqaij
	\item seqbaij
\end{enumerate}

\section{Matrix Collection}
Set of widely used set of sparse matrix benchmarks collected from a wide range of applications.
Link:
\url{https://sparse.tamu.edu/} \\
	
\section{Free Linear Algebra Software}
\url{http://www.netlib.org/utk/people/JackDongarra/la-sw.html} \\
\url{http://www.netlib.org/lapack/\#_presentation} \\
	
CG using Jacobi preconditioner:\\
\url{https://www.npmjs.com/package/conjugate-gradient} \\
	
Iterative methods library:\\
\url{https://math.nist.gov/iml++/ } \\
\section{CG}
Github repository: High Performance Computing Conjugate Gradients: The original Mantevo miniapp \\
\url{https://github.com/Mantevo/HPCCG} 
\begin{itemize}
\item Generates a 27-point finite difference matrix with a user-prescribed sub-block size on each processor
\item Code compiles with MPI support and can be run on one or more processors
\item Input: nx, ny, nz are the number of nodes in the x, y and z dimension respectively on a each processor. The global grid dimensions will be nx, ny and numproc * nz. In other words, the domains are stacked in the z direction.
\end{itemize}
\section{B-CG}
Paper: An Implementation of Block Conjugate Gradient Algorithm on CPU-GPU Processors \\
Link to paper: \\ 
\url{https://ieeexplore.ieee.org/stamp/stamp.jsp?tp=&arnumber=7017966}
\end{document}