\documentclass[]{scrartcl}
\usepackage{hyperref}
\usepackage{enumitem}

\begin{document}
\section{MATRIX MULTIPLICATION}
Given $A=(a_{ij})$ and $B=(b_{ij})$ are square matrices of order n, then $C=(c_{ij})=AB$ is also a square matrix of order n, and $c_{ij}$ is obtained by taking the dot product of the $ith$ row of $A$ with the $jth$ column of $B$. In other words, $c_{ij}=a_{i0}b_{0j}+a_{i1}b_{1j}+...+a_{i,n-1}b_{n-1,j}$.  \\

Matrix Multiplication Algorithm:

\begin{enumerate}
	\item for each row of C
	{\setlength\itemindent{25pt} \item for each column of C }
	{\setlength\itemindent{50pt} \item C[row][column]=0.0; }
	{\setlength\itemindent{50pt} \item for each element of this row of A }
	{\setlength\itemindent{75pt} \item Add A[row][element]*B[element][column] to C[row][column]}	
\end{enumerate}

We can see that a straightforward parallel implementation of matrix multiplication is costly. If we suppose that the number of processes is the same as the order of the matrices; i.e, $p=n$, and suppose that we have distributed the matrices by rows. So process 0 is assigned row 0 of $A$, $B$, and $C$; process 1 is assigned row 1 of $A$, $B$, and $C$; etc. Then in order to form the dot product of the $ith$ row of $A$ with $jth$ column of $B$, we will need to form the dot product of the $jth$ column with every row of $A$. So we will have to carry out an allgather (gathers data from all tasks and distribute the combined data to all tasks) rather than a gather (gathers together values from a group of processes), and we will have to do this for every column of B. This involves a lot of expensive communication. Similarly, an algorithm that distributes the matrices by columns will involve large amount of communication. In order to reduce communication, most parallel matrix multiplication algorithms use a checkerboard distribution of the matrices. The matrices are then seen as a grid. Instead of assigning entire rows or columns to each process, we assign small submatrices. For example, if we have a $4\times4$ matrix, we can assign the elements of the matrix in the following way to 4 processes: 

\begin{center}
	\begin{tabular}{|c|c|} 
		\hline
		Process 0 & Process 1 \\
		$a_{00} \quad a_{01}$ &   $a_{02} \quad a_{03}$ \\
		$a_{10} \quad a_{11}$ &	  $a_{12} \quad a_{13}$ \\ 
		Process 2 & Process 3 \\
		$a_{20} \quad a_{21}$ &	  $a_{22} \quad a_{23}$ \\
		$a_{30} \quad a_{31}$ &	  $a_{32} \quad a_{33}$ \\ 
		\hline
	\end{tabular}
\end{center}

\subsection{Fox's Algorithm}
Assume that the matrices have order n, and the number of processes, $p$, equals $n^{2}$. Then a checkerboard mapping assigns $a_{ij}$, $b_{ij}$, and $c_{ij}$ to process $(i,j$). \\ Fox's algorithm proceeds in n stages: one stage for each term $a_{ik}b_{kj}$ in the dot product $c_{ij}=a_{i0}b_{0j}+a_{i1}b_{1j}+...+a_{i,n-1}b_{n-1,j}$. \\

\begin{enumerate}
	\item Initial stage: each process multiplies the diagonal entry of $A$ in its process row by its element of $B$: \newline
	Stage 0 on process $(i,j)$: $c_{ij}$=$a_{ii}b_{ij}$.
	\item Next stage: each process multiplies the element immediately to the right of the diagonal of $A$ (in its process row) by the process of $B$ directly beneath its own element of $B$: \newline
	Stage 1 on process $(i,j)$: $c_{ij}$=$c_{ij}$+$a_{i,i+1}b_{i+1,j}$. 
	\item In general, during the $kth$ stage, each process multiplies the element $k$ columns to the right of the diagonal of $A$ by the element $k$ rows below its own element of $B$: \newline
	Stage $k$ on process $(i,j)$: $c_{ij}$=$c_{ij}$+$a_{i,i+k}b_{i+k,j}$.
	\item We may encounter out of range subscript when adding $k$ to a row or column. Instead, w define $\bar{k}$=$(i+k)$ mod $n$. Hence, for  stage $k$ on process $(i,j)$: $c_{ij}$=$c_{ij}$+$a_{i,\bar{k}}b_{\bar{k},j}$. \newline
	$c_{ij}$ will be computed as follows: \newline
	$c_{ij}=a_{ii}b_{ij}+a_{i,i+i}b_{i+1}{j}+...+a_{i,n-1}b_{n-1,j}+a_{i0}b_{0j}+...+a_{i,i-1}b_{i-1,j}$.
\end{enumerate} 
We will store sub-matrices rather than matrix elements on each process, provided that  the sub-matrices can be multiplied as needed and the number of process rows or process columns, $\sqrt{p}$, evenly divides $n$. Given this assumption, each process is assigned a square $\frac{n}{\sqrt{p}}\times\frac{n}{\sqrt{p}}$ sub-matrix of each of the three matrices. Define $q=\sqrt{p}$ and $\bar{n}=\frac{n}{\sqrt{p}}$.Denote the sub-matrices $A_{ij}$, $B_{ij}$, and $C_{ij}$ by $A[i,j]$, $B[i,j]$, and $C[i,j]$, the \textbf{pseudo-code of Fox's algorithm is the following:}
\begin{enumerate}
	\item my process row = i, my process column = j
	\item $q=\sqrt{p}$
	\item dest = ( $(i-1)$ mod $q$, $j$ )
	\item source = ( $(i+1)$ mod $q$, $j$ )
	\item for (stage = 0$;$ stage $<q;$ stage++) 
	{\setlength\itemindent{25pt} \item $\bar{k}=(i+stage)$mod$q$}
	{\setlength\itemindent{25pt} \item Broadcast $A[i,\bar{k}]$ across process row $i$}
	{\setlength\itemindent{25pt} \item $C[i,j]=C[i,j]+A[i,\bar{k}]*B[\bar{k},j]$}
	{\setlength\itemindent{25pt} \item Send $B[\bar{k},j]$ to dest; Receive}
	{\setlength\itemindent{25pt} \item $B[(\bar{k}+1)$ mod $q$, $j$] from source}
\end{enumerate}

\textbf{The algorithm can also be written in the following way:}

\begin{enumerate}
	\item for (stage = 0$;$ stage $<q;$ stage++) 
	{\setlength\itemindent{25pt} \item Choose a sub-matrix of $A$ of each row of processes}
	{\setlength\itemindent{25pt} \item In each row of processes broadcast the sub-matrix chosen in that row to the other processes in that row}
	{\setlength\itemindent{25pt} \item On each process, multiply the newly received sub-matrix of $A$ by the sub-matrix of $B$ currently residing on the process}
	{\setlength\itemindent{25pt} \item On each process, send the sub-matrix of $B$ to the process directly above. (On processes in the first row, send the sub-matrix to the last row)}
\end{enumerate}

\subsection{Example Applied to 2$\times$2 Matrices}
Assume we have $n^{2}$ processes, one for each element in A, B, and C. \\
\textbf{Stage 0:}
\begin{enumerate}
	\item $A_{i,i}$ on process $P_{i,j}$, so broadcast the diagonal elements of $A$ across the rows ($A_{ii}\rightarrow P{ij}$). This will place $A_{0,0}$ on each $P_{0,j}$ and $A_{1,1}$ on each $P_{1,j}$. \\
	The $A$ elements on the $P$ matrix will be:
	\begin{center}
		\begin{tabular}{|c|c|} 
			\hline
			$A_{00}$ & $A_{00}$ \\
			$A_{11}$ & $A_{11}$ \\
			\hline
		\end{tabular}
	\end{center}
	\item $B_{i,j}$ on process $P_{i,j}$, so broadcast $B$ across the rows ($B_{ij}\rightarrow P_{ij}$). The $A$ and $B$ values on the $P$ matrix will be:
	\begin{center}
		\begin{tabular}{|c|c|} 
			\hline
			$A_{00}$ & $A_{00}$ \\
			$B_{00}$ & $B_{01}$ \\
			$A_{11}$ & $A_{11}$ \\
			$B_{10}$ & $B_{11}$ \\
			\hline
		\end{tabular}
	\end{center}
	\item Compute $C_{ij}=AB$ for each processor:
	\begin{center}
		\begin{tabular}{|c|c|} 
			\hline
			$A_{00}$ 			  & $A_{00}$ \\
			$B_{00}$   			  & $B_{01}$ \\
			$C_{00}=A_{00}B_{00}$ & $C_{01}=A_{00}B_{01}$ \\
			$A_{11}$              & $A_{11}$ \\
			$B_{10}$     		  & $B_{11}$ \\
			$C_{10}=A_{11}B_{10}$ & $C_{11}=A_{11}B_{11}$ \\
			\hline
		\end{tabular}
	\end{center}
	Now we are ready for the next stage. In this stage, we broadcast the next column (mod n) of $A$ across the processes and shift up (mod n) the $B$ value.
\end{enumerate}

\textbf{Stage 1:}
\begin{enumerate}
	\item The next column is $A_{0,1}$ for the $1^{st}$ row and $A_{1,0}$ for the $2^{nd}$ row. Broadcast the next $A$ across the rows:
		\begin{center}
		\begin{tabular}{|c|c|} 
			\hline
			$A_{01}$ 			  & $A_{01}$ \\
			$B_{00}$   			  & $B_{01}$ \\
			$C_{00}=A_{00}B_{00}$ & $C_{01}=A_{00}B_{01}$ \\
			$A_{10}$              & $A_{10}$ \\
			$B_{10}$     		  & $B_{11}$ \\
			$C_{10}=A_{11}B_{10}$ & $C_{11}=A_{11}B_{11}$ \\
			\hline
		\end{tabular}
	\end{center}
	\item Shift the $B$ values up. $B_{1,0}$ moves up from process $P_{1,0}$ to process $P_{0,0}$ and $B_{0,0}$ moves up (mod n) from $P_{0,0}$ to $P_{1,0}$. Similarly for $B_{1,1}$ and $B_{0,1}$.
	\begin{center}
		\begin{tabular}{|c|c|} 
			\hline
			$A_{01}$ 			  & $A_{01}$ \\
			$B_{10}$   			  & $B_{11}$ \\
			$C_{00}=A_{00}B_{00}$ & $C_{01}=A_{00}B_{01}$ \\
			$A_{10}$              & $A_{10}$ \\
			$B_{00}$     		  & $B_{01}$ \\
			$C_{10}=A_{11}B_{10}$ & $C_{11}=A_{11}B_{11}$ \\
			\hline
		\end{tabular}
	\end{center}
	\item Compute $C_{ij}=AB$ for each process:
		\begin{center}
		\begin{tabular}{|c|c|} 
			\hline
			$A_{01}$ 			  & $A_{01}$ \\
			$B_{10}$   			  & $B_{11}$ \\
			$C_{00}=C_{00}+A_{01}B_{01}$ & $C_{01}=C_{01}+A_{01}B_{11}$ \\
			$A_{10}$              & $A_{10}$ \\
			$B_{00}$     		  & $B_{01}$ \\
			$C_{10}=C_{10}+A_{10}B_{00}$ & $C_{11}=C_{11}+A_{10}B_{01}$ \\
			\hline
		\end{tabular}
	\end{center} 
\end{enumerate}

\subsection{Cannon's Algorithm}
In the following algorithm, all the subscripts are modulo $n$.
\begin{enumerate}
	\item Initially each $p_{i,j}$ has $a_{i,j}$ and $b_{i,j}$
	\item Align elements $a_{i,j}$ and $b_{i,j}$ by reordering so that $a_{i,j+1}$ and $b_{i+j,j}$ are on $p_{i,j}$
	\item Each $p_{i,j}$ computes $c_{i,j}=a_{i,j+1}\times b_{i+j,j}$ ($a_{i,j+i}$ and $b_{i+j,j}$ are local on $p_{i,j}$)
	\item For $k=1$ to $n-1$:
	{\setlength\itemindent{25pt} \item Rotate $A$ left by one column}
	{\setlength\itemindent{25pt} \item Rotate $B$ up by one row}
	{\setlength\itemindent{25pt} \item Each $p_{i,j}$ computes $c_{i,j}=c_{i,j}+a_{i,j+i+k} \times b_{i+j+k,j}$ ($a_{i,j+i+k}$ and $b_{i+j+k,j}$ are local on $p_{i,j}$ after $k$ iterations).}
	
\end{enumerate}

\end{document}
